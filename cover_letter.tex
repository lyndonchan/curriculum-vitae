\documentclass[12pt]{cv_style}
\usepackage{marvosym}
\usepackage{xcolor}
\usepackage{tcolorbox}
\usepackage{fontawesome}
\renewcommand{\familydefault}{\sfdefault}

\setlength{\parindent}{4em}
\setlength{\parskip}{1em}

\begin{document}

\font\titlefont=cmss12 at 36pt

%% Personal Information %%
\info{Lyndon Chan}{647-330-1294}{lyndon.chan@mail.utoronto.ca}{lyndonchan.github.io}{github.com/lyndonchan}

\begin{flushleft}
%% Date %%
August 13, 2019

%% Salutation %%
Dr. Vanessa Allen\\
Chief, Medical Microbiology\\
Public Health Ontario Laboratory\\
661 University Avenue\\
Toronto, ON M5G 1L7

Dear Dr. Vanessa Allen,

My name is Lyndon Chan and I am an MASc candidate in Electrical Engineering at the University of Toronto planning to defend my thesis in September 2019; my research focuses on weakly-supervised semantic segmentation (WSSS) in histopathology images. With my extensive practical and theoretical expertise in image recognition and segmentation, I would be highly qualified for the ``Computer Vision and Deep Learning Internship''. Furthermore, I am excited by the prospect of applying my skills to a clinically relevant problem with potential to improve human lives.

%From the job posting, my understanding is that the desired candidate should have both practical expertise in developing and implementing deep computer vision models for object detection and segmentation, as well as theoretical understanding of how these methods work.

I have been working intensively with deep convolutional neural networks for more than two years now since the summer of 2017. Initially, I experimented with developing a novel convolutional neural network architecture with fixed filter bases to speed up training and introduce rotation invariance. This was conducted in MatConvNet and involved intensive study of the mathematics of convolutional neural networks at each layer.

Later on, I was involved with annotating a large dataset of histopathological images collected with Huron Digital Pathology and St. Michael's Hospital by histological tissue type �- this dataset was introduced in CVPR 2019 as a poster presentation. To validate the annotation quality, I implemented standard CNN models for tissue type prediction in Keras (TensorFlow backend), using image preprocessing, cyclical learning rate, and hierarchical binary relevance to improve performance.

Afterwards, I developed a novel WSSS method for segmenting the histopathology images in the CVPR 2019 dataset, called HistoSegNet -- this will be published soon as a poster presentation in ICCV 2019. I demonstrated that with simple modifications, applying Grad-CAM with CRF outperforms retrained state-of-the-art WSSS methods (in TensorFlow).

I not only excel in implementing and applying deep convolutional models to solve practical computer vision problems but also have extensive theoretical knowledge and am able to both propose and convey novel research ideas. I believe that my education and research expertise would make me an excellent fit for your open position.

Sincerely,

Lyndon Chan

\end{flushleft}
\end{document}