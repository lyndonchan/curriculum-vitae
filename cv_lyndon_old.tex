\documentclass[12pt]{cv_style}
\usepackage{marvosym}
\usepackage{xcolor}
\usepackage{tcolorbox}
\usepackage{fontawesome}
\renewcommand{\familydefault}{\sfdefault}

\begin{document}

\font\titlefont=cmss12 at 36pt

%% Personal Information %%
\info{Lyndon Chan}{647-330-1294}{lyndon.chan@mail.utoronto.ca}{lyndonchan.github.io}{github.com/lyndonchan}

%% Education %%
\sectionheading{Education}
\flushleft\desc{M.A.Sc., Electrical Engineering}{2017-present (Dec. 2019 completion)}{University of Toronto}{Toronto, Ontario, CANADA}
\begin{ditem}
	\item \textsc{Advisors:} Konstantinos Plataniotis \& Parham Aarabi
	\item \textsc{Thesis:} \textit{Weakly-Supervised Semantic Segmentation in the Multi-Class Setting across Different Image Domains} (defended Oct. 4, 2019)
	\item \textsc{Research Milestones:}
	\begin{ditem}
		\item \textsc{ICCV 2019} (Mar. 2019): \emph{HistoSegNet: Semantic Segmentation of Histological Tissue Type
in Whole Slide Images} (\href{http://openaccess.thecvf.com/content_ICCV_2019/papers/Chan_HistoSegNet_Semantic_Segmentation_of_Histological_Tissue_Type_in_Whole_Slide_ICCV_2019_paper.pdf}{paper})
		\item \textsc{CVPR 2019} (Nov. 2018): \emph{Atlas of Digital Pathology: A Generalized Hierarchical Histological Tissue Type-Annotated Database for Deep Learning} (\href{http://openaccess.thecvf.com/content_CVPR_2019/html/Hosseini_Atlas_of_Digital_Pathology_A_Generalized_Hierarchical_Histological_Tissue_Type-Annotated_CVPR_2019_paper.html}{paper})
		\item \textsc{IEEE TMI} (Nov. 2018): \emph{Focus Quality Assessment of High-Throughput Whole Slide Imaging in Digital Pathology} (\href{https://arxiv.org/abs/1811.06038}{paper})
		\item \textsc{2018 EngSci Machine Intelligence Bootcamp} (Sep. 2018): poster on \emph{Automated Abnormality Detection in Histopathological Images with Deep Learning}
	\end{ditem}
\end{ditem}
\medspace
\desc{B.A.Sc., Electrical Engineering (GPA 3.64 / 4.0, 17th of 129)}{2012-2017}{University of Toronto}{Toronto, Ontario, CANADA}
\begin{ditem}
	\item \textsc{Focus Areas:} "Control, Communications \& Signal Processing", "Analog \& Digital Electronics", "Software"
	\item \textsc{Capstone Project:} \textit{DARI: Depth-variable Augmented Reality Interface}	
\end{ditem}

%% Skills %%
\sectionheading{Skills}
\begin{ditem}
	\item \textbf{Programming Languages (most to least proficient):} Python (Keras, TensorFlow, Caffe), MATLAB, C/C++, Java, Ruby, R
	\item \textbf{Software:} \LaTeX, Windows Shell, Wiki Markup, Jupyter Notebook
	\item \textbf{Languages:} English (native), Cantonese (fluent), Mandarin (conversational)
\end{ditem}

%% Interests %%
\sectionheading{Interests}
\begin{ditem}
	\item \textsc{Research Interests:} Weakly-Supervised Semantic Segmentation (WSSS), Computational Pathology, Computer Vision, Computer-aided Diagnosis (CADx), Abnormality Detection
	\item \textsc{Other Interests:} Coding useful tools, Podcasting, Blogging, Teaching, Reading (history, philosophy), Music, Cooking, Translation, Hiking, Running, Swimming
\end{ditem}

\newpage
%% Research %%
\sectionheading{Research}
\desc{Master's Student Research Assistant}{Sep. 2017-present}{University of Toronto (Multimedia Lab)}{Toronto, Ontario, CANADA}
\textsc{Supervisors:} Konstantinos Plataniotis \& Parham Aarabi
\begin{ditem}
	\item Developed weakly-supervised semantic segmentation for histological tissue type in digital pathology, annotated digital pathology patches by histological tissue type to build deep learning dataset
	\item Drafted a study of mathematical derivations of CNN forward and backpropagation
	\item Attended CVPR2019 in Long Beach, California with funding from competitive SGS Conference Grant, will serve as CVPR2020 student reviewer
	\item Administered lab research meetings, interviewed summer student researchers
\end{ditem}

%
\desc{Undergraduate Student Research Assistant}{May 2017-Aug. 2017}{University of Toronto (Multimedia Lab)}{Toronto, Ontario, CANADA}
\textsc{Supervisors:} Mahdi S. Hosseini, Konstantinos Plataniotis
\begin{ditem}
	\item Devised novel image recognition method using a network of fixed convolutional kernels with maximally-polynomial frequency response
\end{ditem}
%
\desc{Interim Engineering Intern}{May 2015-Aug. 2016}{Qualcomm Canada}{Markham, Ontario, CANADA}
\begin{ditem}
	\item Software development: built regression test frameworks for optical flow, cadence detection, deinterlacing, image compression
	\item Other work: performed subjective image quality assessment, administered and operated camera calibration lab \& mechanical camera testbed, competed in two internal Qualcomm HackMobile hackathons
\end{ditem}
%
\desc{Undergraduate Visiting Research Intern}{Jun.-Aug. 2014}{Hong Kong University of Science and Technology (Human Language Technology Centre)}{\\Clear Water Bay, New Territories, HONG KONG}
\textsc{Supervisors:} SU Dan, Pascale Fung
\begin{ditem}
	\item (1) Unsupervised clustering of user personalities by nationality from OkCupid
	\item (2) Song popularity prediction from user mentions on Sina Weibo posts
\end{ditem}
%
%\desc{High School Summer Research Assistant}{Jul.-Aug. 2012}{Sunnybrook Research Institute (Focused Ultrasound Group)}{Toronto, Ontario, CANADA}
%\textsc{Supervisors:} Mathew Carias, Dr. Kullervo Hynynen
%\begin{ditem}
	%\item Built and optimized ultrasound transducer catheters for therapeutic cardiac ablation through destructive testing
%\end{ditem}

%% Teaching %%
\sectionheading{Teaching}
\desc{ECE462: Multimedia Systems (Head Lab TA)}{Jan.-Apr. 2018}{University of Toronto}{Toronto, Ontario, CANADA}
\textsc{Instructor:} Dimitrios Hatzinakos\\
\begin{ditem}
	\item Responsible for designing and marking eight undergraduate lab assignments and four quizzes on image processing and compression, compiled student material for CEAB review, was awarded ECE Student Club Teaching Assistant Award
\end{ditem}
\desc{ECE1512: Digital Image Processing and Applications (Head TA)}{Sep.-Dec. 2019}{University of Toronto}{Toronto, Ontario, CANADA}
\textsc{Instructor:} Konstantinos Plataniotis\\
\begin{ditem}
	\item Responsible for designing and marking two graduate-level assignments and a final project on XAI and CNN classification
\end{ditem}

%% Publications %%
%\sectionheading{Publications}
%\begin{ditem}
	%\item
%\end{ditem}
%
\end{document}
